\chapter{Introdução}

Nesta palestra, vou concentrar"-me nos aspetos essenciais da meditação a
andar abordando as questões `Como?', `Quando?', `Onde?' e `Porquê?'.
Pretendo incluir instruções práticas sobre os aspetos técnicos da
meditação a andar e instruções para cultivar qualidades mentais que
levam à concentração, introspeção e sabedoria através da atividade
física da meditação a andar.

O Buddha enfatizou o desenvolvimento da plena atenção nas quatro
principais posturas: de pé, sentado, deitado e a andar (DN 22, MN 10). Ele
encorajou"-nos a estarmos atentos em todas estas posturas, para termos
plena consciência e reconhecimento daquilo que estamos a fazer em
qualquer posição.

Se lerem acerca da vida dos monges e monjas no tempo de Buddha,
perceberão que muitos alcançaram certos estágios de iluminação enquanto
praticavam meditação a andar. Meditação a andar chama"-se \emph{caṅkama}
em Pali. É uma atividade que permite focar e concentrar a mente
ou desenvolver conhecimento e sabedoria introspetiva.

Algumas pessoas sentem"-se naturalmente mais atraídas pela meditação a
andar, por a considerarem mais fácil e natural do que a meditação
sentada. Quando se sentam, sentem"-se aborrecidas, tensas ou distraem"-se
facilmente. As suas mentes não se acalmam. Se este é o vosso caso, não
persistam; façam alguma coisa nova e experimentem mudar de postura.

Façam alguma coisa diferente, experimentem a meditação em pé ou tentem a
meditação a andar. Esta nova postura de meditação poderá tornar"-se numa
forma arguta de focar a mente. Todas as quatro posturas de meditação são
apenas técnicas, métodos para desenvolver e treinar a mente.

Experimentem desenvolver a meditação a andar e talvez possam começar a
aperceber"-se dos seus benefícios. Na Tradição da Floresta, no nordeste
da Tailândia, há uma grande ênfase na meditação a andar. Muitos dos
monges desta tradição praticam"-na durante longos períodos, como uma
forma de desenvolver concentração. Por vezes chegam a praticar até dez
ou quinze horas por dia!

O falecido monge Ajahn Singtong praticava tanto a meditação a
andar que chegava a criar um sulco no caminho de meditação. Ele chegava
a caminhar até quinze horas por dia, tornando o arenoso caminho que
usava, côncavo! Outro Monge, Ajahn Kum Dtum, praticava meditação
a andar tão assiduamente que à noite nem se preocupava em entrar na sua
cabana. Quando se sentia realmente cansado de caminhar durante todo o
dia e toda a noite, dormia mesmo ali, no chão, utilizando as mãos como
almofada. Adormecia com plena atenção tendo determinado levantar"-se
assim que acordasse. Assim que acordava, continuava a andar novamente.
Basicamente vivia no carreiro de meditação. Ajahn Kum Dtum
alcançou rapidamente resultados na sua prática.

No Ocidente não se costuma falar tanto da meditação a andar, por isso eu
gostaria de descrever o processo e recomendá"-la como complemento à
meditação sentada. Espero que estas instruções vos ajudem a desenvolver
o vosso repertório de técnicas de meditação -- tanto em meditação formal
como na vida quotidiana. Como grande parte da vida é passada a andar, se
souberem como estar conscientes disso, o simples facto de andarem em
casa pode tornar"-se num exercício de meditação.

\section{Os Cinco Benefícios da Meditação a Andar}

O Buddha falou sobre cinco benefícios da meditação a andar (AN. III. 29). Neste
\emph{Sutta}, ordenou"-os da seguinte forma: desenvolve a resistência para
caminhar longas distâncias, desenvolve vigor, previne doenças, ajuda a digestão
após a refeição e a concentração adquirida na meditação a andar é duradoura.

\begin{siderule-quote}
  1. Desenvolve resistência para caminhar longas distâncias
\end{siderule-quote}

O primeiro benefício da meditação a andar consiste no desenvolvimento de
resistência para caminhar longas distâncias. Isto era particularmente
importante na época de Buddha, visto a maioria das pessoas
deslocar"-se a pé. O próprio Buddha deslocava"-se regularmente a
pé, de local para local, chegando a caminhar até dezasseis quilómetros
por dia. Desta forma, ele recomendava a meditação a andar como forma de
desenvolver uma condição física saudável e resistência para percorrer
longas distâncias.

Ainda hoje, os monges da floresta fazem longas caminhadas, às quais se
chama \emph{Thudong} (\emph{dhutanga}). Agarram nas suas tigelas de
oferendas e nos seus mantos e partem à procura de lugares isolados para
meditar. Como preparação antes de partirem, aumentam progressivamente as
horas de prática da meditação a andar até um mínimo de cinco ou seis
horas por dia, com o intuito de desenvolverem aptidão física e
resiliência. Se andarem cerca de quatro a cinco quilómetros por hora e
praticarem cinco horas de meditação a andar por dia, o número de
quilómetros vai"-se acumulando.

\begin{siderule-quote}
  2. Desenvolve vigor
\end{siderule-quote}

O desenvolvimento de vigor, especialmente para superar sonolência, é o
segundo benefício da meditação a andar. Enquanto praticam meditação
sentada, as pessoas notam que existe tendência para entrar em estados
tranquilos, mas se estiverem demasiado tranquilos, sem estarem atentos,
podem começar a adormecer, chegando mesmo a ressonar! O tempo passa
rapidamente, e apesar de ser apaziguador, não existe claridade ou
consciência. Sem plena atenção nem consciência, a meditação pode
tornar"-se aborrecida pois é dominada por preguiça e sonolência.
Desenvolver a meditação a andar pode contrariar esta tendência.

Como exemplo, Ajahn Chah recomendava que, uma vez por semana,
ficássemos acordados toda a noite, meditando e caminhando. Normalmente
ficamos bastante sonolentos à uma ou duas da madrugada, por isso,
Ajahn Chah recomendava fazermos meditação a andar de costas, para
contrariar o sono. Não se adormece a andar de costas!

Quando estava no Mosteiro \emph{Bodhinyana}, no oeste da Austrália,
lembro"-me de sair uma manhã cedo, por volta das cinco horas, para
praticar meditação a andar. Vi um dos leigos que tinha passado o Retiro
das Chuvas (vassa) no mosteiro, a fazer um esforço enorme para superar a
sonolência. Ele estava a praticar meditação a andar em cima de um muro
com dois metros de altura, que existia em frente ao mosteiro -- andava
muito atentamente para trás e para a frente em cima do muro! Eu estava
um pouco preocupado que ele pudesse cair e magoar"-se. Contudo, ele
estava a fazer um grande esforço para estar atento a cada passo,
desenvolvendo um elevado estado de alerta, esforço e empenho para
superar a sonolência.

\begin{siderule-quote}
  3. Previne doenças
\end{siderule-quote}

O Buddha disse que a meditação a andar mantém"-nos saudáveis. É o
terceiro benefício. Todos temos consciência que andar é considerado uma
ótima forma de exercício físico. Hoje em dia até ouvimos falar sobre
``\emph{power walking}'' (forma de exercício físico em que se caminha de
maneira rápida e vigorosa). Aqui, estamos a falar de ``\emph{power
meditation}'', ou seja, desenvolver a meditação a andar como exercício
físico e também mental. Desta forma, caminhar não é só uma boa forma de
exercício físico, mas também nos ajuda a cultivar a mente. No entanto,
para usufruirmos destes dois benefícios, temos que estar atentos ao
processo de caminhar, em vez de deixarmos a mente dispersar"-se ao pensar
noutras coisas.

\begin{siderule-quote}
  4. Facilita a digestão após a refeição
\end{siderule-quote}

O quarto benefício da meditação a andar é ajudar a digestão. Para os
monges, que apenas têm uma refeição por dia, este facto é
particularmente importante. Depois de uma refeição, o sangue flui para o
estômago, distanciando"-se do cérebro, o que pode provocar sonolência. Os
Monges da Floresta salientam que, após uma refeição, deve"-se praticar
algumas horas de meditação a andar para facilitar a digestão. Para os
praticantes leigos aplica"-se o mesmo. Depois de uma refeição pesada, em
vez de fazer"-se uma sesta, deve"-se sair e praticar uma hora de meditação
a andar. Ajuda o bem"-estar físico e é uma boa oportunidade de exercitar
a mente.

\begin{siderule-quote}
  5. A concentração adquirida na meditação a andar é duradoura
\end{siderule-quote}

O quinto benefício da meditação a andar é que a concentração gerada por
esta prática mantém"-se por muito tempo. A postura em andamento é na
realidade uma postura pouco refinada em comparação com a postura sentada
no que se refere à atividade envolvida. Enquanto sentados, é fácil
mantermos essa posição; temos os olhos fechados, por isso não estamos
sujeitos a estímulos visuais e não estamos ocupados com o movimento do
corpo. O mesmo é verdade para as posturas de pé e deitado pois estas
também não implicam movimento.

Enquanto caminhamos há muita estimulação sensorial devido ao movimento
do corpo e (de) /a/ estarmos a ver para onde nos dirigimos (estimulação
visual). Portanto, se conseguirmos concentrar a mente enquanto andamos e
recebemos todos estes estímulos sensoriais, então, quando mudamos desta
postura para uma mais refinada, será mais fácil manter a concentração.
Isto é, quando nos sentamos, a energia mental e o poder de concentração
irão facilmente ser transferidos para esta postura mais refinada.

Pelo contrário, se apenas desenvolvemos a concentração na postura
sentada, quando nos levantamos e iniciarmos movimentos corporais mais
evidentes, como o andar, é difícil mantermos o estado de concentração.
Isto acontece porque estamos a transitar de posturas mais refinadas para
menos refinadas. Assim, a meditação a andar pode ajudar a desenvolver
energia, clareza mental e uma concentração que continua noutras
posturas, menos ativas, de meditação.

