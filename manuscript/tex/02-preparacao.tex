% Was: Preparação Para a Prática da Meditação a Andar

\chapter{Preparação}

\begin{siderule-quote}
  Encontrar um local adequado.
\end{siderule-quote}

O local, onde o Senhor Buddha praticava a meditação a andar depois de
atingir a Iluminação, em \emph{Bodhgaya}, ainda existe nos dias de hoje. O
trajeto onde ele praticava esta forma de meditação tinha dezassete passos de
comprimento. Atualmente, os Monges da Floresta tendem a fazer trajetos mais
compridos. Podem ter até trinta passos. Os principiantes podem achar esta
distância um pouco longa, pois ainda não têm a concentração desenvolvida. Quando
chegam ao fim do trajeto, a mente pode já ter dado a volta ao mundo e
regressado. Lembrem"-se que caminhar é estimulante e que inicialmente a mente
tende a dispersar"-se. Normalmente, é melhor os principiantes começarem com um
trajeto mais curto: quinze passos seria uma distância ideal.

Se praticarem meditação ao ar livre, tentem fazê"-lo num local isolado
onde não se distraiam, nem sejam facilmente perturbados. É aconselhável
encontrarem um local para a praticar que seja ligeiramente abrigado.
Poderá ser uma distração caminhar numa zona aberta com paisagem, pois a
mente poderá ser atraída pelo cenário. Uma zona abrigada é especialmente
adequada para pessoas que têm tendência para pensar muito, ajudando"-as a
acalmar a mente (Vsm, III, 103). Se praticarem num local fechado, a
mente tende a voltar"-se para dentro, para o próprio ser, ao encontro da
paz.

\begin{siderule-quote}
  Preparar o corpo e a mente.
\end{siderule-quote}

Quando tiverem escolhido o trajeto adequado, coloquem"-se num dos
extremos do trajeto. Mantenham"-se direitos. Coloquem à frente do vosso
corpo a mão esquerda e por cima dela a mão direita. Não andem com as
mãos atrás das costas. Lembro"-me de um mestre de meditação, que quando
visitou o mosteiro e ao ver um dos visitantes a andar para trás e para a
frente com as mãos atrás das costas, fez o seguinte comentário: ``Ele
não está a praticar meditação a andar, está a passear''. O mestre fez
esta observação porque reparou que não havia uma determinação clara de
focar a mente na meditação a andar, na qual se coloca as mãos à frente
do corpo para distinguir o que estamos a fazer de um simples passeio.

O objetivo primário da prática é desenvolver \emph{samādhi}, e isto
requer esforço focalizado. A palavra em Pali, \emph{samādhi},
significa focar a mente, ou desenvolver a mente até esta se encontrar
unificada através de níveis graduais de atenção e concentração. Para
focarmos a mente necessitamos ser diligentes e determinados. Em primeiro
lugar, é necessário um certo grau de compostura física e mental.
Começamos esta compostura unindo as mãos à nossa frente, tranquilamente.
Uma postura calma do corpo ajuda a compor a mente. Assim, com o corpo
tranquilo, devem permanecer quietos e trazer a consciência e a atenção
para o corpo. Depois, levantem as mãos conjuntamente em \emph{añjali},
um gesto de respeito no qual une"-se verticalmente as palmas das mãos em
frente do peito, e com os olhos fechados reflitam durante alguns minutos
nas qualidades do Buddha, do Dhamma e do Sangha
(\emph{Buddhanussati}, \emph{Dhammanussati} e \emph{Sanghanussati}).

Podem contemplar o facto de terem tomado refúgio no Buddha,
aquele que é Sábio, aquele que Vê e Conhece, o Desperto, o Iluminado.
Reflitam por alguns minutos, com o vosso coração, nas qualidades do
Buddha. Depois recordem"-se do Dhamma -- a Verdade que
estão a tentar realizar e desenvolver ao praticarem meditação a andar.
Finalmente, tragam à mente o Sangha -- especialmente os
Despertos, que realizaram a verdade praticando meditação. Depois ponham
as mãos em baixo à vossa frente e determinem mentalmente durante quanto
tempo irão praticar meditação a andar, seja meia hora, uma hora ou mais.
Independentemente do tempo que determinaram para a prática,
mantenham"-no. Deste modo, nesta fase inicial da meditação, estão a
alimentar a mente com entusiasmo, inspiração e confiança.

É importante lembrarem"-se de olhar para o chão, cerca de um metro e meio
à vossa frente. Mantenham o olhar fixo, não se deixem distrair, seja o
que for que aconteça à vossa volta. Estejam conscientes do que sentem na
planta dos pés, e desenvolvendo assim uma atenção mais refinada e
sabendo claramente que estão a andar, enquanto andam.

