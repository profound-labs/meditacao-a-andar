\chapter{Objetos De Meditação Para Meditação a Andar}

O \emph{Buddha} mencionou quarenta objetos de meditação diferentes nos
seus ensinamentos (Vsm, III, 104), muitos dos quais podem ser utilizados
na meditação a andar. Contudo, uns são mais adequados que outros. Irei
examinar aqui alguns desses objetos de meditação, começando por aqueles
utilizados com mais frequência.

\section{Consciência da postura}

Neste método, enquanto caminhamos dirigimos toda a atenção para a planta
dos pés, reparando nas sensações conforme elas surgem e cessam (isto é,
presumindo que caminham descalços como faz a maioria dos monges, embora
se necessário possam usar sapatos com sola fina). Assim que começam a
andar, a sensação muda. Quando levantam e pousam o pé, pondo-o em
contacto novamente com o solo, surge uma nova sensação. Estejam
conscientes desta sensação assim que ela surge na planta do pé. Quando o
pé se levanta de novo, tomem mentalmente nota da nova sensação, assim
que ela surge. Sempre que levantam e pousam cada pé, estejam conscientes
das sensações. A cada novo passo, novas sensações surgem e velhas
sensações cessam. Elas devem ser reconhecidas conscientemente. A cada
passo, novas sensações são experienciadas -- uma sensação surge, uma
sensação cessa; uma sensação surge, uma sensação cessa.

Com este método, dirigimos a atenção para a sensação de andar, para cada
passo dado, para \emph{vedanā} (sensações agradáveis, desagradáveis ou
neutras). Estamos conscientes de qualquer tipo de \emph{vedanā} que
surge na planta dos pés. Quando estamos de pé, existe uma sensação de
contacto com o solo. Este contacto pode originar dor, calor ou outro
tipo de sensações. Dirigimos a nossa atenção para estas sensações,
reconhecendo-as. Quando levantamos um pé, a sensação muda assim que
perdemos o contacto com o solo. Quando baixamos o pé, assim que este
entra de novo em contacto com o solo, uma nova sensação surge. Enquanto
andamos, as sensações mudam constantemente, umas cessam e novas surgem.
Plenamente atentos, reparem no aparecimento e desaparecimento das
sensações na sola dos pés enquanto os levantam e pousam no chão. Assim
toda a atenção é dirigida apenas para as sensações que surgem enquanto
caminhamos.

Alguma vez repararam nas sensações que surgem nos pés enquanto caminham?
Acontecem sempre que andamos, mas normalmente não reparamos nestas
coisas subtis durante a nossa vida. Enquanto caminhamos a mente
normalmente encontra-se noutro lugar qualquer. A meditação a andar é uma
forma de simplificar o que estamos a fazer enquanto o fazemos. Trazemos
a mente para o 'aqui e agora', e assim estamos conscientes de estar a
andar, enquanto andamos. Ao reconhecermos apenas as sensações que surgem
e cessam, a mente fica tranquila, tornando tudo mais simples.

A que velocidade se deve caminhar? \emph{Ajahn Chah} recomendava que
andássemos naturalmente, nem muito depressa, nem muito devagar. Se
caminharem muito depressa, poderá ser difícil concentrarem-se nas
sensações que surgem e cessam. Podem precisar de caminhar mais
lentamente. Por outro lado, há outras pessoas que podem precisar de
caminhar mais depressa. Depende de indivíduo para indivíduo. Encontrem o
vosso próprio ritmo, aquele que funcionar melhor, seja ele qual for.
Podem começar com um ritmo lento e irem aumentando gradualmente, até
atingirem o ritmo normal de andamento.

Se a vossa atenção é fraca (isto é, se a mente dispersa-se facilmente),
então andem muito devagar, até conseguirem estar no momento presente a
cada passo dado. Comecem por estabelecer plena atenção, no início do
trajeto. Quando chegarem a meio do trajeto perguntem, "Onde está a minha
mente? Está atenta às sensações na planta dos pés? Estou a notar o
contacto aqui e agora, no momento presente?" Se a mente estiver a
vaguear, então tragam-na de volta, reparando novamente nas sensações na
planta dos pés, e continuem a caminhar.

Quando chegarem ao fim do trajeto voltem-se devagar e restabeleçam plena
atenção novamente. Onde está a mente? Está a aperceber-se das sensações
na planta dos pés? Ou vagueou? A mente tende a dispersar-se para outros
lugares, indo atrás de pensamentos de ansiedade, medo, felicidade,
tristeza, preocupações, dúvidas, prazeres, frustrações e muitos outros
que possam surgir. Se não tiverem a atenção plenamente estabelecida no
objeto de meditação, restabeleçam-na antes de recomeçarem a andar.
Restabeleçam a mente no simples ato de andar e voltem para trás
caminhando até ao fim do trajeto. Quando chegarem a meio deverão
reparar, "Agora estou a meio caminho" e confirmem novamente para se
certificarem que a mente está focada no objeto de meditação. Depois,
quando chegarem ao final do trajeto reparem mentalmente, "Onde está a
mente?" Desta forma, andam para trás e para a frente, plenamente
conscientes das sensações que vão surgindo e cessando. Enquanto
caminham, restabeleçam constantemente a atenção, trazendo a mente de
novo para o interior, retomando a atenção plena, estando conscientes e
reconhecendo as sensações a cada momento, assim que surgem e cessam.

Enquanto mantemos plena atenção nas sensações que sentimos na planta dos
pés, notamos que a mente se distrai menos, que não tem tanta tendência a
distrair-se com as coisas que estão a acontecer à volta. Ficamos mais
calmos e a mente fica tranquila. Quando a mente fica calma e tranquila,
irão reparar que a postura de meditação a andar se torna numa atividade
demasiado agitada para uma mente com estas qualidades. Vão querer estar
quietos. Então parem e permitam à mente experienciar essa calma e
tranquilidade. Este estado é conhecido como \emph{passaddhi}, que é um
dos fatores da Iluminação.

Se enquanto andarem, a mente entrar num estado muito refinado, poderão
sentir que é impossível continuar. Para andar é necessário a vontade de
se moverem, e possivelmente a vossa mente encontra-se demasiado focada
no objeto de meditação para continuar. Então parem de andar e pratiquem
meditação em pé. Meditação tem a ver com o trabalho da mente e não com
uma postura específica. A postura física utilizada é apenas uma forma de
potencializar o trabalho da mente.

Concentração e tranquilidade funcionam em conjunto com a atenção plena.
Juntamente com a energia, investigação do \emph{Dhamma}, alegria e
equanimidade, estes são os "Sete Fatores da Iluminação." Quando
meditamos e a mente está tranquila, e é devido a esta mesma
tranquilidade que surgem os estados de alegria, êxtase e beatitude. O
Buddha disse que a beatitude da paz é a felicidade suprema (MN, I, 454),
e uma mente concentrada experiencia essa paz. Esta paz pode ser
experienciada nas nossas vidas.

Tendo desenvolvido a prática da meditação a andar num contexto formal,
então quando estivermos a andar na nossa vida diária, a ir às compras, a
ir de um quarto para o outro, ou mesmo a andar até a casa de banho,
podemos praticar meditação a andar. Podemos estar conscientes de estar a
andar, simplesmente tendo isso presente. As nossas mentes podem estar
calmas e em paz. Esta é uma forma de desenvolvermos tranquilidade e
concentração nas nossas vidas diárias.

\section{Da meditação sentada para a meditação a andar}

Se a mente ficar tranquila com um determinado objeto de meditação
enquanto praticam meditação sentada, usem o mesmo objeto na meditação a
andar. Contudo, com alguns objetos de meditação subtis, tais como a
respiração, primeiro a mente deverá ter atingido um determinado grau de
estabilidade nesse estado de calma. Se a mente ainda não estiver calma e
começarem a meditação a andar focando-se na respiração, vai ser difícil,
pois a respiração é um objeto muito subtil. Geralmente é melhor começar
com um objeto de meditação que seja facilmente reconhecido, como as
sensações que surgem na planta dos pés.

Para fazer a transição da meditação sentada para a meditação a andar, há
muitos objetos de meditação que podem ser utilizados, como por exemplo,
os quatro estados sublimes da mente: Bondade, Compaixão, Alegria
Empática e Equanimidade.

Enquanto caminharem para a frente e para trás, desenvolvam pensamentos
expansivos de bondade, "Possam todos os seres ser felizes, possam todos
os seres estar em paz, possam todos os seres estar livres de todo o
sofrimento". Pode-se usar a postura de meditação a andar como um
complemento da meditação sentada, desenvolvendo a meditação com o mesmo
objeto mas numa postura diferente.

\section{Escolher um mantra}

Se sentirem sonolência enquanto praticam meditação a andar, então devem
estimular a mente, em vez de a acalmarem, usem um mantra para que ela
fique mais focada e desperta. Usem um mantra como \emph{Buddho},
repetindo contínua e silenciosamente a palavra para vós mesmos. Se a
mente continuar a vaguear, então comecem a dizer \emph{Buddho}
rapidamente e andem para trás e para a frente depressa. Enquanto andam,
recitem a palavra \emph{Buddho}, \emph{Buddho}, \emph{Buddho}. Desta
forma a mente poderá focar-se rapidamente.

Quando \emph{Tan Ajahn Mun}, um famoso mestre de meditação da Tradição
da Floresta, esteve no norte da Tailândia, com as tribos das montanhas,
eles não sabiam nada acerca de meditação ou de monges que meditavam.
Contudo, as pessoas dessas tribos são muito curiosas. Quando o viram a
caminhar para trás e para a frente, percorrendo o seu trajeto,
seguiram-no em fila. Quando ele chegou ao fim do trajeto e se voltou,
toda a tribo estava ali de pé!

Repararam que \emph{Ajahn Mun} estava a andar para trás e para a frente
com os olhos postos no chão e presumiram que andava à procura de alguma
coisa. Perguntaram, "O que anda à procura Venerável Senhor? Podemos
ajudá-lo a encontrar?" \emph{Tan Ajahn Mun} respondeu astutamente,
"Estou à procura de \emph{Buddho}, o \emph{Buddha} no coração. Vocês
podem ajudar-me a encontrá-lo andando para trás e para a frente, fazendo
os vossos próprios trajetos, procurando o \emph{Buddha}." E com esta
instrução simples e bonita muitos deles começaram a meditar, e \emph{Tan
Ajahn Mun} disse que obtiveram excelentes resultados.

\section{Contemplar as coisas como elas são}

Investigar o \emph{Dhamma} (\emph{Dhammavicaya}) é um dos fatores de
Iluminação, e é um tipo de contemplação acerca dos ensinamentos e das
leis da natureza que pode ser utilizado quando andamos para trás e para
a frente, no caminho de meditação a andar. Isto não significa que
estejamos apenas a especular ou a pensar sobre qualquer coisa antiga.
Pelo contrário, é uma reflexão e contemplação constante sobre a Verdade
(\emph{Dhamma}).

Investigar a impermanência: podemos, por exemplo, contemplar a
impermanência observando o processo de mudança e perceber que todas as
coisas estão sujeitas à mudança. Desenvolvemos uma perceção clara do
surgir e do cessar de toda a experiência. "A Vida" é um processo
contínuo de surgimento e cessação e toda a experiência condicionada está
sujeita a esta lei da natureza. Contemplando esta Verdade compreendemos
as características da existência. Vemos que todas as coisas estão
sujeitas à mudança. Todas as coisas são insatisfatórias. Todas as coisas
são impessoais. Podemos investigar estas características fundamentais da
natureza na prática da meditação a andar.

\section{Relembrando a generosidade e a virtude}

O Buddha insistiu continuamente na importância da generosidade (It, 26)
e da virtude (SN, V, 354). Enquanto praticamos meditação a andar,
podemos refletir sobre a nossa virtude ou atos de generosidade.
Caminhando para trás e para a frente, perguntem-se, "Que atos de bondade
fiz hoje?"

Um mestre de meditação com o qual pratiquei, comentava frequentemente
que uma das razões porque os meditadores não conseguiam estar em paz é o
facto de durante o dia não terem praticado suficientemente a bondade. A
bondade é uma almofada de tranquilidade, uma base para termos paz. Se a
praticarmos durante o dia -- termos dito uma palavra amável, termos
feito uma boa ação, termos sido generosos ou termos tido compaixão --
então a mente experienciará alegria e êxtase. Estes atos de bondade e a
felicidade daí resultante são as condições que determinarão a
concentração e a paz. O poder da bondade e da generosidade conduz à
felicidade, e esta felicidade saudável é a fundação da concentração e da
sabedoria.

Quando a mente está inquieta, agitada, com raiva ou frustrada,
recordarmos as nossas boas ações é um objeto de meditação apropriado. Se
a mente não está em paz, então lembrem-se das vossas ações bondosas. O
objetivo não é alimentar o ego, mas reconhecer o poder da bondade e da
moralidade. Atos de bondade, virtude e generosidade trazem alegria à
mente, e a alegria é um fator de Iluminação (SN, V, 68).

Relembrar atos de generosidade, refletir nos benefícios de dar, recordar
a vossa virtude, contemplar a pureza da não-violência, a pureza da
honestidade, a pureza de relações sexuais corretas, a pureza da
veracidade, a pureza da mente sem confusão quando se evitam
intoxicantes; quando praticamos meditação a andar todas estas memórias
podem servir de objeto de meditação.

\section{Relembrando a natureza do corpo}

Podemos também meditar sobre a morte ou na natureza não atraente do
corpo, a partir das contemplações \emph{Asubha} -- de cadáveres em
diversos estados de decomposição. Podemos visualizar este corpo separado
em partes, tal como um estudante de medicina disseca um corpo.
``Descascamos'' a pele e ``vemos'' o que está por baixo, as camadas de
carne, os tendões, os ossos, os órgãos. Podemos mentalmente remover cada
um dos órgãos do corpo para que possa ser investigado e compreendido. De
que é feito o corpo? Que partes o compõem? Isto sou eu? Isto é
permanente? É digno de ser chamado eu?

O corpo é apenas um aspeto da natureza, como uma árvore ou uma nuvem --
não é diferente. O problema fundamental é a nossa identificação com ele:
quando a mente se apega à ideia de que este corpo é o meu corpo, quando
se deleita com este corpo e quando se deleita com outros corpos; quando
pensa ' Isto sou eu. Isto sou eu próprio. Isto pertence-me'.

Podemos desafiar esta identificação com o corpo através da contemplação
e da investigação. Tomamos como objeto os ossos deste corpo. Quando
praticarmos meditação a andar, visualizamos um osso e compreendemos a
sua substância, vendo-o fragmentar-se e voltar ao elemento terra. O osso
é formado por cálcio que é absorvido pelo corpo através do consumo de
vegetais e de matéria animal. Ele vem da terra. Os químicos juntam-se
para formar o osso, e eventualmente esse osso voltará à terra.

Cálcio é apenas cálcio, não tem a qualidade de ser o meu cálcio ou o de
outra pessoa qualquer. Terra apenas volta para a terra, cada elemento
volta à sua forma natural. ``Isto não sou eu, isto não é meu, isto não é
digno de ser chamado eu''. Meditamos sobre os ossos decompondo-os nos
seus elementos e devolvendo-os à terra. Voltamos a fazê-lo novamente,
decompondo-os e devolvendo-os. Continuamos este processo mental até que
surja uma compreensão clara.

Se estiverem a meditar no corpo e ainda não tiverem decomposto
totalmente o objeto de meditação nos quatro elementos (terra, ar, fogo e
água) e o reconstituírem novamente, então o trabalho da meditação ainda
não está concluído. O trabalho não está feito. Continuem. Continuem a
andar. Andem para trás e para a frente até estarem aptos a estabelecer a
perceção mental de ver \emph{asubha} no \emph{subha,} ou seja, verem a
não-beleza, a falta de encanto e o não atraente no que à partida
assumimos como belo, encantador e atraente. Com o objetivo de o vermos
como realmente é, decompomos este corpo e devolvemo-lo aos seus
elementos naturais.

O treino da mente na investigação da natureza conduz à sabedoria. Ao
repetir o exercício de decompor o corpo nos seus quatro elementos --
terra, ar, fogo e água -- a mente vê e compreende que este corpo não sou
eu, não é meu, não sou eu próprio. Ela vê que os quatro elementos que
constituem o corpo são apenas aspetos da natureza. É a mente que se
apega à ideia de que este corpo sou eu próprio. Então, desafiamos esse
apego; não o aceitamos cegamente, pois é esse apego que nos causa todo o
sofrimento.

\section{Outras contemplações}

Outro objeto de meditação que o \emph{Buddha} recomendou é a reflexão
sobre a paz e a sua natureza (Vsm, 197). Ainda outro é ponderar sobre as
qualidades da Iluminação. Alternativamente, podemos andar para trás e
para a frente refletindo nas qualidades do \emph{Buddha}, nas qualidades
do \emph{Dhamma} ou nas qualidades do \emph{Sangha}. Ou podemos trazer à
memória seres divinos (\emph{devas}) e as qualidades necessárias para
nos tornarmos um ser divino (Vsm, III, 105).

\section{Utilizar a contemplação sabiamente}

No repertório da meditação budista há imensos objetos de meditação. O
vosso objeto de meditação deve ser escolhido cuidadosamente. Selecionem
um que estimule a mente quando esta necessita ser estimulada, ou que a
pacifique quando ela necessite de ser acalmada. No entanto, quando
utilizamos estas contemplações na prática de meditação é necessária
alguma prudência para que a mente não entre em pensamentos
especulativos, divagando. Isto é muito fácil de acontecer. Temos que
estar muito atentos e repararmos no início, no meio e no final do
trajeto: "Tenho realmente o meu objeto de meditação presente ou estou a
pensar noutra coisa qualquer?" Se estiver a andar para trás e para a
frente, no trajeto de meditação, durante quatro horas, mas se estiver
atento e consciente do que está a fazer apenas durante um minuto dessas
quatro horas, terei praticado apenas um minuto de meditação.

Lembrem-se de que não é a quantidade de horas de meditação que
interessa, mas sim a qualidade. Se a vossa mente estiver a vaguear por
outro lado qualquer enquanto anda, então não estão a meditar. Não estão
a meditar no sentido em que o Buddha usou a palavra meditação, como
\emph{Bhāvanā} ou desenvolvimento mental (AN, III, 125-127). O mais
importante não é a quantidade de horas que cada um medita, mas sim a
qualidade da mente.

